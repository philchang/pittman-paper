%\documentstyle[emulateapj,apjfonts,epsfig]{article}
\documentclass{emulateapj}
\usepackage{amsmath, mathtools}

%\documentclass[12pt,preprint]{aastex}
%\usepackage{emulateapj5,apjfonts}

\newcommand{\EXO}{\mbox{EXO 0748-676}}
\newcommand{\keV}{${\rm keV}$}
\newcommand{\Fe}{${\rm Fe} \ $}
\newcommand{\Mn}{${\rm Mn} \ $}
\newcommand{\Cr}{${\rm Cr} \ $}
\newcommand{\V}{${\rm V} \ $}
\newcommand{\Ti}{${\rm Ti} \ $}
\newcommand{\Sc}{${\rm Sc} \ $}
\newcommand{\Ca}{${\rm Ca} \ $}
\newcommand{\be}{\begin{equation}}
\newcommand{\ee}{\end{equation}}
\newcommand       \etaeff       {\eta}
\newcommand       \tff          {\tau_{\rm ff}}
\newcommand       \tdyn         {\tau_{\rm dyn}}


\shorttitle{Neutron Star Redshift Measurements} 
\shortauthors{Bildsten, Chang and Paerels}

\begin{document}

\title{NEED TITLE}

\author{some people}


                        %%%%%%%%%%%%%%%%%%%%%%%%
                        %       abstract       %
                        %%%%%%%%%%%%%%%%%%%%%%%%
\begin{abstract}

write this

\end{abstract}

\keywords{diffusion -- nuclear reactions -- 
stars: abundances, surface -- stars: neutron -- X-rays: binaries, bursts}

\section{Introduction}

\section{Analytic Expectations}

Recently, Murray \& Chang (2014) considered the spherical collapse of isothermal turbulently supported gas accounting for adiabatic heating from compression of the turbulence. In particular,  {\bf SHOW EQUATIONS}
Their basic results are that the run of density asymptotes to 
\be
\rho(r,t)=
\begin{dcases}
\rho(r_0)\left({r\over r_0}\right)^{-3/2}, & r<r_*\\
\rho(R,t)\left({r\over R}\right)^{-k_\rho}, \ k_\rho\approx1.6-1.8 & r>r_*.
\end{dcases}
\ee
Since $k_\rho$ is fairly close to $1.5$ at all radii, taking $r_0=R$
is a fair approximation. 
The infall velocity 
%
\be
u_r(r,t)=
\begin{dcases}
-\Gamma\sqrt{GM_*(t)\over r}, \sim r^{-1/2} & r<r_*\\
-\Gamma\sqrt{GM(r,t)\over r} \sim r^{0.2} & r>r_*,
\end{dcases}
\ee
%
where $\Gamma \approx 0.7$ at small radii, and $\Gamma\approx 1.0$ at
large radii. 

The turbulent velocity 
%
\be
v_T(r,t)=
\begin{dcases}
{1\over 2\etaeff}\Gamma\sqrt{GM_*(t)\over r}, \sim r^{-1/2} & r<r_*\\
{1.2\over \etaeff}\Gamma\sqrt{GM(r,t)\over r} \sim r^{0.2} & r>r_*,
\end{dcases}
\ee
%

The stellar mass increases quadratically with time
%
\be  %$
M_*(t)=\phi M_{\rm cl}\left({t-t_*\over \tff}\right)^2.
\ee  %$
%

The mass accretion rate 
%
\be
\dot M(r,t)=
\begin{dcases}
4\pi R^2\rho(R)u_r(r,t), \sim t\,r^{0} & r<r_*\\
4\pi R^2\rho(R)u_r(r,t) \sim t^0\,r^{0.2} & r>r_*.
\end{dcases}
\ee
%


\section{Detailed Simulations of Turbulent Collapse}

We use the adaptive mesh refinement code FLASH ver. 4.0.1 (Fryxell et al.
2000
; Dubey
et al.
2008
) to model isothermal, self-gravitating, hydrodynamic turbulence on isothermal gas with three-dimensional (3D),
periodic grids and 10 levels of refinement on a root grid of $128^3$, giving an effective resolution of $128K^3$.  
Self-gravity is computed with a multi-grid Poisson solver (see Ricker
2008), coupled with a fast-Fourier transform solution on the root grid.

To initialize our simulations, we drive turbulence by applying a large scale ($1 \le k \le 2$) solenoidal 
acceleration field as a momentum and energy source term.  We apply this field in the absence of gravity and sink particle formation for 3 dynamical times until a statistical steady state is reached.
{\bf WHY JUST SOLENOIDAL}

This fully developed turbulent state is the initial condition to which we add self-gravity and sink particle formation for
our star formation experiments. Sink particles are formed as described in Lee et al. (2014) and described briefly below.  

The gravitational force is computed differently than the gas. Sink particle-sink particle forces are computed via direct N-body calculation. While sink particle-gas and gas-sink particle is computed via the multi-grid Poisson solver.  As a result of these additional computations, two large scale gravity solutions must be found per timestep as oppose to one.  This allows to avoid the computationally expensive task of computing gas-sink particle forces via direct summation. 

We have also implemented a new algorithm for mesh refinement in these simulations.  As gas collapse under self gravity, certain regions rapidly increase in density.  These regions are refine when the Truelove criterion ($\lambda \le 4 \Delta x$) is met. This corresponds to a condition on the density 
\begin{equation}
refinement criteria
\end{equation}
which when met causes the local grid to be refined by a factor of 2 provided that the maximum refinement level is not reached.

When the Truelove criterion is met at highest refinement level, the excess mass in a cell is transferred either to a newly created sink particle or to a sink particle whose accretion radius includes the cell.  This behavior is the same as in Lee et al. (2014), albeit at a much higher resolution.  We should also note that like Lee et al. (2014), our sink particle creation prescription is different from the prescription of Federrath et al (???) where additional checks need to be performed.


\section{Numerical Results}

To initialize our simulation, we begin by applying solenoidal stirring forces to the uniform, periodic simulation volume at the root grid resolution of $128^3$ until statistical equilibrium is reached (approximately three crossing times).  After this equilibrium is established, we then apply self gravity, adaptive mesh refinement, and sink particle creation.  In Figure \ref{fig:entire projection} we show a projection along the x-axis of the entire simulation volume that has up to 10 levels of refinement, giving an effective resolution of $128\,{\rm K}^3$. Regions that are highly refined are the densest regions which are smoother than the low-density more pixelated regions.
\begin{figure}
\plotone{frame0241.png}
\caption{Projection along the x-axis of the entire simulation volume. The root grid is $128^3$ with up to 10 levels of refinement, giving an effective resolution of $128\,{\rm K}^3$. This snapshot is taken at ??. \label{fig:entire projection}}
\end{figure}

The high density regions appear to be organized along filaments.  These filaments span most of the simulation size, and have a width of 0.5 to 1 pc.  Moreover, these filaments appear to flow into large clumps.  This is in line with previous work including Lee et al. 2014. These clumpy regions have the highest densities and, hence, are prone to fulfill the criterion for star particle formation.  

At the instance in time shown in Figure \ref{fig:entire projection}, a total of five star particles have been formed.  However, these star particles are not formed in isolation.  Rather they congregate in two star forming regions shown in Figure \ref{fig:star forming regions}.  In these regions, the sink particles that are formed lay within 0.1 pc of one another, which reflects that the highest density regions are concentrated toward the center.  Visually, the character of accretion at the this outer scale appear to flow along filaments, though previous work (Lee et al. 2014) has shown that the majority of accretion is spherical.  We discuss this more quantitatively below {\bf INCLUDE A STUDY OF LARGE SCALE ACCRETION}


\begin{figure*}
\plottwo{Disk_0132_0000.png}{Disk_0132_0002.png}
\caption{Cut of the two star forming regions.\label{fig:star forming regions}}
\end{figure*}

The first family of three star particles is shown in Figure \ref{fig:snapshots} and the second family of two star particles is shown in Figure \ref{fig:snapshots 2}.  We plot these particles and their surrounding $4\times 10^{-2}$ pc region as slices perpendicular to the local angular momentum vector.  We determine the local angular momentum vector by taking a $0.01$ pc region about the star particle to calculate the angular momentum vector.  It is clear from these plots that the clumpy filamentary nature of the local star forming environment extends down to much smaller scales than we have been able to probe in previous work (Lee et al. 2014).  

\begin{figure*}
\plottwo{Average_5_013.pdf}{Average_5_24.pdf}
\plottwo{Average_02_013.pdf}{Average_02_24.pdf}
\caption{Here we show the mass accretion rates as a function of density.  The upper figures are displaying rates corresponding to a region of $0.5$ pc around the particle, while the lower figures are showing a radius of $0.02$ pc.  The two plots on the left show the region with three particles forming, and the plots on the right show the region with two particles forming.  The more massive region with three particles indicates the mass is coming in at a density value roughly three times the average density, and the curves are similar between the smaller and larger radii.  The graph for the less massive region on the right shows a distinctly separate profile between the larger and the smaller radius.   \label{fig:accretion average rho}}
\end{figure*}

In prior work, we examined the geometry of accretion, i.e., does the accretion proceed along filament or over 4$\pi$ steradian, by plotting the cumulative accretion rate as a function of density. In Figure \ref{fig:accretion average rho}, we plot the cumulative accretion rate as a function of density, normalized to the average density in shells of radii 0.5 (upper plots) and 0.02 pc (lower plots).  Plots on the left are associated with the more massive star forming region, where 3 star particles reside and plots on the right are associated with the less massive star forming region, where two star particles reside.  The density where 50\% of the accretion is accounted for occurs at $\rho/\bar{\rho} \approx 3$ for the more massive star forming region and appears to be independent of whether the shell has a radius of 0.5 or 0.02 pc.  On the other hand, the less massive star forming region has half of the accretion account for at $\rho/\bar{\rho} \approx 10$ at r=0.5 pc.  This suggests that the accretion is directed along filaments.  This filamentary nature of accretion does not maintain itself to lower radii, where at 0.02 pc, the 50\% point in cumulatiove accertion occurs at $\rho/\bar{\rho} \approx 1$, which suggest accretion along $4\pi$ steradian.
{\bf INCLUDE ACCRETION TOPOLOGY STUDY HERE.}

At small radii (similar to $10^{-2}$ pc), a protostellar disk appears in all five snapshots. The radial size of these disks varies. These disks also show a coherent sense of rotation about a common axis, e.g., no counter-rotating disks. In addition, some of the snapshots show multiple high density regions.  This could be a result of accretion of high density clumps or fragmentation in the disk or streams.  

We also include a zoom-out picture of the same region, which is the same for all three stars (lower right plot in Figure \ref{fig:snapshots}, bottom plot in Figure \ref{fig:snapshots 2}).  While this is not unexpected as star formation is clustered, the large scale dynamics are determined by the larger region.  We should note that these star particles are within about 0.1 pc of each other in their respective families.  

{\bf ARE THESE DISK GRAVITATIONALLY STABLE?}

\begin{figure*}
\plottwo{movie_disk_0132_0001.png}{movie_disk_0132_0003.png}
\plottwo{movie_disk_0132_0000.png}{Disk_0132_0000.png}
\caption{The upper frames, and the bottom left frame show slices along the angular momentum axis of three local particles in one of the regions demonstrating significant star formation.  Each of the zoomed-in slices have a radius of $0.02$ pc.  To determine the angular momentum axis, we take a $0.01$ pc region about the densest local point and calculate the angular momentum vector.  We then take a slice in the plane normal to the angular momentum vector to plot the regions around these dense points.  The bottom right frame shows the zoomed-out region of formation, with a radius of $1.5$ pc.    \label{fig:snapshots}}
\end{figure*}

\begin{figure*}
\includegraphics[width=0.48\textwidth]{movie_disk_0132_0002.png}
\includegraphics[width=0.48\textwidth]{movie_disk_0132_0004.png}
\centering{\includegraphics[width=0.48\textwidth]{Disk_0132_0002.png}}
\caption{The upper frames show the zoomed-in slices along the angular momentum axis of the local particles in the second region of significant star production.  Again, each of the zoomed-in slices have a radius of $0.02$ pc.  The bottom frame shows the zoomed out region, with a radius of $1.5$ pc.  \label{fig:snapshots 2}}
\end{figure*}

\begin{figure*}
\includegraphics[width=0.48\textwidth]{geometry_0.pdf}
\includegraphics[width=0.48\textwidth]{geometry_1.pdf}
\centering{\includegraphics[width=0.48\textwidth]{geometry_3.pdf}}
\caption{$geometry_0$, $geometry_1$, $geometry_3$   
\label{fig:geometry}}
\end{figure*}

\begin{figure*}
\plottwo{geometry_2.pdf}{geometry_4.pdf}
\caption{$Geometry_2$ and $geometry_4$  
\label{fig:geometry 2}}
\end{figure*}


 


\subsection{Radial Profile of Collapse}

These simulations display much more complicated dynamics than is captured by the simple analytic theory of Murray and Chang 2014.  In spite of this, we compare radially averaged profiles around these high density points to their analytic theory.  In the left panel of Figure \ref{fig:good_rho_example}, we show the radial density profile around one such high density point. As this figure shows, the radial density profile follows a $r^{-3/2}$ power law in agreement with analytic expections.  This power-law breaks down at small radii, but this is not unexpected as this radii corresponds to the presents of a rotationally support disk.  Such good agreement is not universal for all high density points as shown in the right panel of Figure \ref{fig:bad_rho_example}.  Here the radial density profile follows a $r^{-2}$ power law.  
%\begin{figure*}
%\plottwo{Track_0256_000_rho.png}{Track_0132_000_rho.png}
%\caption{Density as a function of radial position from maximum density point.  The density reaches a maximum of $10^6$ times mean density at $r \approx 10^{-3}$ pc and declines to approximately mean density at $r\approx 3$ pc.  The three straight solid lines are $r^{-3/2}$, $r^{-2}$, and $r^{-5/2}$ power laws. \label{fig:good_rho_example}
%{bf\ This discusses left panel only, should we move the discussion outside the caption?}}
%\end{figure*}


However, to make a comparison of the average behavior of high density star forming regions, we average over a number 

What plots should we show?
-velocities vs r
-angular momentum vs r
-mass vs r
-mdot vs r

In Figure \ref{fig:vr_example}, we plot $v_r$ (blue line) and $v_{\rm rms}$ (green line) as a function of $r$.  In the left panel, the radial velocity curve is sporadic, but the general trend is that $v_r$ declines in toward $0.1$ pc, but then rises inward of this point.  Plotted in red is a $r^{-1/2}$ power law and this inward rise appears to follow this general trend.  The turbulent velocity is smoother, but it follows the general trend of the $v_r$ curve in that it falls slowly inward toward $]\approx 0.1 $ pc before rising inward of that, roughly tracking the $r^{-1/2}$ power law.  These general trends are in rough agreement with our analytic expectations.  In the right panel, the radial velocity curve is smoother and increases inwards towards $0.01$ pc, tracking the $r^{-1/2}$ power law fairly closely.  This radial velocity begins to fall inward of $0.01$ pc, possibly due to rotational support from the angular momentum of a disk.  The turbulent velocity in the right panel seems to roughly follow an inverse relationship with the radial velocity.  As with the left panel, however, the turbulent velocity is significantly less volatile than the radial velocity and varies only slightly with $r$.   These expectations are in line with the analytic results of Murray and Chang (2014), albeit with a somewhat larger scale that they suggest.  
 
%\begin{figure*}
%\plottwo{Track_0256_000_vr.png}{Track_0132_000_vr.png}
%\caption{Radial velocity, $v_r$ (blue line), and turbulent velocity, $v_{\rm rms}$ (green line) as a function of radial position from maximum density point.  A $r^{-1/2}$ power law is shown as a red line. \label{fig:vr_example}}
%\end{figure*}

A hint of why these different behaviors emerge from looking at the $M(r)$.  In Figure \ref{fig:mass_example}, we plot the mass as a function of radius.  Regions where $M(r)$ flattens implies that the mass in this region can be treated using a point mass approximation.  Here it is apparent that the regions that can be treated a point mass corresponds to regions in where the density follows a $r^{-3/2}$ power law and the radial and turbulent velocities follow a $r^{-1/2}$ power law.  This is due to mass concentration at smaller radii as evident by the sudden drop in $M(r)$ at $r \approx 10^{-2} - 1$ pc.  The flattening of $M(r)$ on larger scales is interesting as this is due to the entire mass of the forming star cluster as opposed to the individual stars that MC14 originally envisioned.  Hence the entire mass of the star cluster contributes to the large scale dynamics of turbulent collapse.    

%\begin{figure*}
%\plottwo{Track_0256_000_mass.png}{Track_0132_000_mass.png}
%\caption{WRITE CAPTION: Mass, $MSun$, as a function of radial position from a local maximum density point.     \label{fig:mass_example}}
%\end{figure*}

\begin{figure*}
\plottwo{Disk_0132_0000.png}{Track_0132_000_rho.png}
\plottwo{Track_0132_000_vr.png}{Track_0132_000_mass.png}
\plottwo{Track_0132_000_mass.png}{pdf_frame_0132_0000.png}
\caption{Tracking 132 000  These plots show various data for one of the particles formed in the simulation.  The upper left graph shows the density profile, with reference lines indicating slopes of -3/2, -2, and -5/2.  The slope between $0.02$pc and $0.4$pc approximately matches the -5/2 reference line.  The plot on the upper right shows the pdf function related to volumes of 1pc, 2pc, and 5pc.  The bottom left shows a plot of the radial velocity and turbulent velocity associated with the particle.  The lower right graph shows the radial density profile of the particle.    \label{fig:132 000 graphs}}
\end{figure*}

\begin{figure*}
\plottwo{Disk_0132_0002.png}{Track_0132_002_rho.png}
\plottwo{Track_0132_002_vr.png} {Track_0132_002_mass.png}
\plottwo{Track_0132_002_mass.png}{pdf_frame_0132_0002.png}
\caption{Tracking 132 002  These plots show the data for a different particle formed in the simulation at the same time as the previous plots.  The upper left graph of the density in the region neatly matches the -3/2 slope.  The pdf graph in the upper right again shows the plots for volumes related to $1$pc, $2$pc, and $5$pc.  The lower left plot demonstrates the loose connection between the radial velocity and the turbulent velocity.  The graph on the lower right again shows the density profile of this second particle.     \label{fig:132_002_graphs}}
\end{figure*}





\begin{figure*}
\plottwo{Track_0256_000_l.png}{Track_0132_000_l.png}
\caption{WRITE CAPTION: Angular momentum as a function of distance from maximum density point.     \label{fig:angular_momentum_example}}
\end{figure*}

%\begin{figure*}
%\plottwo{pdf_frame_0256.png}{pdf_frame_0132.png}
%\caption{Probability density function, $pdf$ as a function of density $rho/rho_0$.  \bf Verify range of slope calculation \label{fig:pdf_example}}
%\end{figure*}

%\begin{figure*}
%\plottwo{Disk_0132_0000.png}{Disk_0132_0004.png}
%\caption{Visualization of density profiles for two distinct particles in tracking1b. \label{fig:rho_example_132}}
%\end{figure*}

\subsection{Average Profiles}

\section{Discussion and Conclusions}

\acknowledgments

Write

\begin{references}

\noindent
Bethe, H.~A. \& Salpeter, E.~E. 1957 \textit{Quantum Mechanics of One-
  and Two-Electron Atoms} (Berlin: Springer-Verlag) (BS) 

\noindent
Bildsten, L., Salpeter, E.~E. \& Wasserman, I. 1992, \apj, 384, 143
(BSW) 


\end{references}

\end{document}




%\documentstyle[emulateapj,apjfonts,epsfig]{article}
\documentclass{emulateapj}
\usepackage{amsmath, mathtools}

%\documentclass[12pt,preprint]{aastex}
%\usepackage{emulateapj5,apjfonts}

\newcommand{\EXO}{\mbox{EXO 0748-676}}
\newcommand{\keV}{${\rm keV}$}
\newcommand{\Fe}{${\rm Fe} \ $}
\newcommand{\Mn}{${\rm Mn} \ $}
\newcommand{\Cr}{${\rm Cr} \ $}
\newcommand{\V}{${\rm V} \ $}
\newcommand{\Ti}{${\rm Ti} \ $}
\newcommand{\Sc}{${\rm Sc} \ $}
\newcommand{\Ca}{${\rm Ca} \ $}
\newcommand{\be}{\begin{equation}}
\newcommand{\ee}{\end{equation}}
\newcommand       \etaeff       {\eta}
\newcommand       \tff          {\tau_{\rm ff}}
\newcommand       \tdyn         {\tau_{\rm dyn}}


\shorttitle{Neutron Star Redshift Measurements} 
\shortauthors{Bildsten, Chang and Paerels}

\begin{document}

\title{NEED TITLE}

\author{some people}


                        %%%%%%%%%%%%%%%%%%%%%%%%
                        %       abstract       %
                        %%%%%%%%%%%%%%%%%%%%%%%%
\begin{abstract}

write this

\end{abstract}

\keywords{diffusion -- nuclear reactions -- 
stars: abundances, surface -- stars: neutron -- X-rays: binaries, bursts}

\section{Introduction}

\section{Analytic Expectations}

Recently, Murray \& Chang (2014) considered the spherical collapse of isothermal turbulently supported gas accounting for adiabatic heating from compression of the turbulence. In particular,  {\bf SHOW EQUATIONS}
Their basic results are that the run of density asymptotes to 
\be
\rho(r,t)=
\begin{dcases}
\rho(r_0)\left({r\over r_0}\right)^{-3/2}, & r<r_*\\
\rho(R,t)\left({r\over R}\right)^{-k_\rho}, \ k_\rho\approx1.6-1.8 & r>r_*.
\end{dcases}
\ee
Since $k_\rho$ is fairly close to $1.5$ at all radii, taking $r_0=R$
is a fair approximation. 
The infall velocity 
%
\be
u_r(r,t)=
\begin{dcases}
-\Gamma\sqrt{GM_*(t)\over r}, \sim r^{-1/2} & r<r_*\\
-\Gamma\sqrt{GM(r,t)\over r} \sim r^{0.2} & r>r_*,
\end{dcases}
\ee
%
where $\Gamma \approx 0.7$ at small radii, and $\Gamma\approx 1.0$ at
large radii. 

The turbulent velocity 
%
\be
v_T(r,t)=
\begin{dcases}
{1\over 2\etaeff}\Gamma\sqrt{GM_*(t)\over r}, \sim r^{-1/2} & r<r_*\\
{1.2\over \etaeff}\Gamma\sqrt{GM(r,t)\over r} \sim r^{0.2} & r>r_*,
\end{dcases}
\ee
%

The stellar mass increases quadratically with time
%
\be  %$
M_*(t)=\phi M_{\rm cl}\left({t-t_*\over \tff}\right)^2.
\ee  %$
%

The mass accretion rate 
%
\be
\dot M(r,t)=
\begin{dcases}
4\pi R^2\rho(R)u_r(r,t), \sim t\,r^{0} & r<r_*\\
4\pi R^2\rho(R)u_r(r,t) \sim t^0\,r^{0.2} & r>r_*.
\end{dcases}
\ee
%


\section{Numerical Results}

We use the adaptive mesh refinement code FLASH ver. 4.0.1 (Fryxell et al.
2000
; Dubey
et al.
2008
) to model isothermal, self-gravitating, hydrodynamic turbulence on isothermal gas with three-dimensional (3D),
periodic grids and 10 levels of refinement on a root grid of $128^3$, giving an effective resolution of $128K^3$.  
Self-gravity is computed with a multi-grid Poisson solver (see Ricker
2008), coupled with a fast-Fourier transform solution on the root grid.

To initialize our simulations, we drive turbulence by applying a large scale ($1 \le k \le 2$) solenoidal 
acceleration field as a momentum and energy source term.  We apply this field in the absence of gravity and sink particle formation for 3 dynamical times until a statistical steady state is reached.
{\bf WHY JUST SOLENOIDAL}

This fully developed turbulent state is the initial condition to which we add self-gravity and sink particle formation for
our star formation experiments. Sink particles are formed as described in Lee et al. (2014). {\bf INCLUDE EQUATION}.  Note that this is different from the sink particle prescription Federrath et al (???) when additional checks need to be performed.

The gravitational force is computed differently than the gas. Sink particle-sink particle forces are computed via direct N-body calculation. While sink particle-gas and gas-sink particle is computed via the multi-grid Poisson solver.  As a result of these additional computations, two large scale gravity solutions must be found per timestep as oppose to one.  This allows to avoid the computationally expensive task of computing gas-sink particle forces via direct summation. 

{\bf include discussion of mesh refinement}

{\bf include discussion of star formation.}

To initialize our simulation, we begin by applying solenoidal stirring forces to the uniform, periodic simulation volume at the root grid resolution of $128^3$ until statistical equilibrium is reached (approximately three crossing times).  After this equilibrium is established, we then apply self gravity, adaptive mesh refinement, and sink particle creation.  In Figure \ref{fig:entire projection} we show a projection along the x-axis of the entire simulation volume that has up to 10 levels of refinement, giving an effective resolution of $128\,{\rm K}^3$. Regions that are highly refined are the densest regions which are smoother than the low-density more pixelated regions.
\begin{figure}
\plotone{frame0241.png}
\caption{Projection along the x-axis of the entire simulation volume. The root grid is $128^3$ with up to 10 levels of refinement, giving an effective resolution of $128\,{\rm K}^3$. This snapshot is taken at ??. \label{fig:entire projection}}
\end{figure}

The high density regions appear to be organized along filaments.  These filaments span most of the simulation size, and have a width of 0.5 to 1 pc.  Moreover, these filaments appear to flow into large clumps.  This is in line with previous work including Lee et al. 2014.  

The simulated volume has a large number of refined regions.  We show four of them in Figure \ref{fig:snapshot} as slices perpendicular to the angular momentum vector.  We determine the angular momentum vector by taking a $0.01$ pc region about the densest local point and calculate the angular momentum vector.  

\begin{figure*}
\plottwo{movie_disk_frame_0259.png}{movie_disk_frame_0257.png}
\plottwo{movie_disk_frame_0256.png}{movie_disk_frame_0272.png}
\caption{Slices along the angular momentum axis of four zoomed in simulation. To determine the angular momentum axis, we take a $0.01$ pc region about the densest local point and calculate the angular momentum vector. We then take a slice in the plane normal to the angular momentum to plot the regions around these dense points. \label{fig:snapshots}}
\end{figure*}


These figures suggest that the accretion can occur by streams or by spherical flows.  However, the degree by which these modes of accretion dominate, is not clear.  At small radii (similar to $10^{-2}$ pc), a protostellar disk appears in all four snapshots. The radial size of these disks varies. These disks also show a coherent sense of rotation about a common axis, e.g., no counter-rotating disks. In addition, some of the snapshots show multiple high density regions.  This could be a result of accretion of high density clumps or fragmentation in the disk or streams.   

These simulations display much more complicated dynamics than is captured by the simple analytic theory of Murray and Chang 2014.  In spite of this, we compare radially averaged profiles around these high density points to their analytic theory.  In the left panel of Figure \ref{fig:good_rho_example}, we show the radial density profile around one such high density point. As this figure shows, the radial density profile follows a $r^{-3/2}$ power law in agreement with analytic expections.  This power-law breaks down at small radii, but this is not unexpected as this radii corresponds to the presents of a rotationally support disk.  Such good agreement is not universal for all high density points as shown in the right panel of Figure \ref{fig:bad_rho_example}.  Here the radial density profile follows a $r^{-2}$ power law.  
\begin{figure*}
\plottwo{Track_0256_000_rho.png}{Track_0132_000_rho.png}
\caption{Density as a function of radial position from maximum density point.  The density reaches a maximum of $10^6$ times mean density at $r \approx 10^{-3}$ pc and decline to approximately mean density at $r\approx 3$ pc.  The three straight solid lines are $r^{-3/2}$, $r^{-2}$, and $r^{-5/2}$ power laws. \label{fig:good_rho_example}}
\end{figure*}


However, to make a comparison of the average behavior of high density star forming regions, we average over a number 

What plots should we show?
-velocities vs r
-angular momentum vs r
-mass vs r
-mdot vs r

In Figure \ref{fig:vr_example}, we plot $v_r$ (blue line) and $v_{\rm rms}$ (green line) as a function of $r$.  In the left panel, the radial velocity curve is sporadic, but the general trend is that $v_r$ declines in toward $0.1$ pc, but then rises inward of this point.  Plotted in red is a $r^{-1/2}$ power law and this inward rise appears to follow this general trend.  The turbulent velocity is smoother, but it follows the general trend of the $v_r$ curve in that it falls slowly inward toward $]\approx 0.1 $ pc before rising inward of that, roughly tracking the $r^{-1/2}$ power law.  These general trends are in rough agreement with our analytic expectations.  

{\bf DISCUSS RIGHT PANEL} 
 
\begin{figure*}
\plottwo{Track_0256_000_vr.png}{Track_0132_000_vr.png}
\caption{Radial velocity, $v_r$ (blue line), and turbulent velocity, $v_{\rm rms}$ (green line) as a function of radial position from maximum density point.  A $r^{-1/2}$ power law is shown as a red line. \label{fig:vr_example}}
\end{figure*}

A hint of why these different behaviors emerge from looking at the $M(r)$.  In Figure \ref{fig:mass_example}, we plot the mass as a function of radius.  Regions where $M(r)$ flattens implies that the mass in this region can be treated using a point mass approximation.  Here it is apparent that the regions that can be treated a point mass corresponds to regions in where the density follows a $r^{-3/2}$ power law and the radial and turbulent velocities follow a $r^{-1/2}$ power law. 

\begin{figure*}
\plottwo{Track_0256_000_mass.png}{Track_0132_000_mass.png}
\caption{WRITE CAPTION: Radial velocity, $v_r$ (blue line), and turbulent velocity, $v_{\rm rms}$ (green line) as a function of radial position from maximum density point.  A $r^{-1/2}$ power law is shown as a red line. \label{fig:mass_example}}
\end{figure*}

\section{Discussion and Conclusions}

\acknowledgments

Write

\begin{references}

\noindent
Bethe, H.~A. \& Salpeter, E.~E. 1957 \textit{Quantum Mechanics of One-
  and Two-Electron Atoms} (Berlin: Springer-Verlag) (BS) 

\noindent
Bildsten, L., Salpeter, E.~E. \& Wasserman, I. 1992, \apj, 384, 143
(BSW) 


\end{references}

\end{document}




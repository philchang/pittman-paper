%\documentstyle[emulateapj,apjfonts,epsfig]{article}
\documentclass{emulateapj}
%\documentclass[12pt,preprint]{aastex}
%\usepackage{emulateapj5,apjfonts}

\newcommand{\EXO}{\mbox{EXO 0748-676}}
\newcommand{\keV}{${\rm keV}$}
\newcommand{\Fe}{${\rm Fe} \ $}
\newcommand{\Mn}{${\rm Mn} \ $}
\newcommand{\Cr}{${\rm Cr} \ $}
\newcommand{\V}{${\rm V} \ $}
\newcommand{\Ti}{${\rm Ti} \ $}
\newcommand{\Sc}{${\rm Sc} \ $}
\newcommand{\Ca}{${\rm Ca} \ $}

\shorttitle{Neutron Star Redshift Measurements} 
\shortauthors{Bildsten, Chang and Paerels}

\begin{document}

\title{NEED TITLE}

\author{Lars Bildsten\altaffilmark{1}, Philip Chang\altaffilmark{2},
and Frits Paerels\altaffilmark{3}}
\altaffiltext{1}{Kavli Institute for Theoretical Physics,  Kohn Hall, University
of California, Santa Barbara, CA 93106, USA; email:
bildsten@kitp.ucsb.edu}
\altaffiltext{2}
{Department of Physics, Broida Hall, University of California,
Santa Barbara, CA 93106; pchang@physics.ucsb.edu}
\altaffiltext{3}
{Columbia Astrophysics Laboratory and Department of Astronomy,
Columbia University, 538 W. 120th St., New York, NY 10027;
frits@astro.columbia.edu}


                        %%%%%%%%%%%%%%%%%%%%%%%%
                        %       abstract       %
                        %%%%%%%%%%%%%%%%%%%%%%%%
\begin{abstract}

write this

\end{abstract}

\keywords{diffusion -- nuclear reactions -- 
stars: abundances, surface -- stars: neutron -- X-rays: binaries, bursts}

\section{Introduction}

\section{Analytic Expectations}
\section{Comparison with Numerical Results}

We use the adaptive mesh refinement code FLASH ver. 4.0.1 (Fryxell et al.
2000
; Dubey
et al.
2008
) to model isothermal, self-gravitating, hydrodynamic turbulence on isothermal gas with three-dimensional (3D),
periodic grids and 10 levels of refinement on a root grid of $128^3$, giving an effective resolution of $128K^3$.  
Self-gravity is computed with a multi-grid Poisson solver (see Ricker
2008), coupled with a fast-Fourier transform solution on the root grid.

To initialize our simulations, we drive turbulence by applying a large scale ($1 \le k \le 2$) solenoidal 
acceleration field as a momentum and energy source term.  We apply this field in the absence of gravity and sink particle formation for 3 dynamical times until a statistical steady state is reached.
{\bf WHY JUST SOLENOIDAL}

This fully developed turbulent state is the initial condition to which we add self-gravity and sink particle formation for
our star formation experiments. Sink particles are formed as described in Lee et al. (2014). {\bf INCLUDE EQUATION}.  Note that this is different from the sink particle prescription Federrath et al (???) when additional checks need to be performed.

The gravitational force is computed differently than the gas. Sink particle-sink particle forces are computed via direct N-body calculation. While sink particle-gas and gas-sink particle is computed via the multi-grid Poisson solver.  As a result of these additional computations, two large scale gravity solutions must be found per timestep as oppose to one.  This allows to avoid the computationally expensive task of computing gas-sink particle forces via direct summation. 


What plots should we show?
-density vs r
-velocities vs r
-angular momentum vs r
-mass vs r

\begin{figure}
\plotone{frame0241.png}
\end{figure}

\begin{figure*}
\plottwo{movie_disk_frame_0259.png}{movie_disk_frame_0257.png}
\plottwo{movie_disk_frame_0256.png}{movie_disk_frame_0272.png}
\end{figure*}

\begin{figure}
\plotone{mass.png}
\end{figure}

\begin{figure}
\plotone{Track_0132_000_rho.png}
\end{figure}

\section{Discussion and Conclusions}

\acknowledgments

Write

\begin{references}

\noindent
Bethe, H.~A. \& Salpeter, E.~E. 1957 \textit{Quantum Mechanics of One-
  and Two-Electron Atoms} (Berlin: Springer-Verlag) (BS) 

\noindent
Bildsten, L., Salpeter, E.~E. \& Wasserman, I. 1992, \apj, 384, 143
(BSW) 


\end{references}

\end{document}



